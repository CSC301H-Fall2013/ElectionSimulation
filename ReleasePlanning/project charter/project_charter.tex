\documentclass{report} 
\begin{document}\begin{center}
\section*{Project Charter\\Team Great Bear\\}
\end{center}
Hypothetically, upon completion of the software, its functionality will be divided as per the permissions of the user(s). We will have defined two different kinds of users: Campaign Master, and Presenter/Player. The Player will have the authority to set the parameters for the overall simulation environment which the Campaign Master will be confined to. Then, the Campaign Master will allocate the resources dedicated by the Player for the simulation of an election. The Campaign Master will choose a type of election to simulate (FPTP or STV etc) and the software will generate the results based on all the federal election data we will have been able to procure. Since the model will function mostly on statistics, which revolves around uncertainty, we can be fairly certain of success if the system predicts the results according to the tendency of the data, or if we the data as a validation set and run a simulation with the same parameters to attempt to generate the same outcome.\\\\
On the presenter's end, success would be if the system properly generated a neat graphical representation of the results, demonstrating the various proportions of the voters. One of the more important features of the software would be to generate nice graphs of the results because the software is politically-motivated; the end result is to argue to some high-authority figures that proportional voting is feasible at least within the city of Toronto, so it would have to be appealing and simple for them to understand and being able to see the proportions of voters should convince them additionally the voice of the voters is vastly under-represented. 
\end{document}







